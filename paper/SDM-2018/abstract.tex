\begin{abstract}
In networking applications, one often wishes to obtain estimates
about the number of objects at different parts of the network
(e.g., the number of cars at an intersection of a road network 
or the number of packets expected to reach a node in a computer network)
by monitoring the traffic in a small number of network nodes or edges. 
We formalize this task by defining the \mcproblem\ problem. 

Given an initial distribution of
items over the nodes of a {\markovchain}, we wish to
estimate the distribution of items at subsequent times. 
We do this by asking a limited number of queries that retrieve, for example,  
how many items transitioned to a specific node
or over a specific edge at a particular time.
We consider different types of queries, each defining a different variant of the {\mcproblem}.
For each variant, we design efficient algorithms for choosing the  queries that make our estimates 
as accurate as possible.
In our experiments with synthetic and real datasets we demonstrate the
efficiency and the efficacy of our algorithms in a variety of settings.

% -- but are
%constrained by the type and number of queries we are allowed to perform.
%The problem consists in choosing the queries so as to 
%make as accurate estimates as possible.
%% The problem finds natural application on {\textit{network traffic estimation}}
%% -- and leads to novel and intuitive definitions of {node} and {edge centrality}
%% on networks.
%
%In this paper, we define variants of the problem for different types 
%of queries, provide efficient algorithms to solve them, and
%showcase the performance of the algorithms on real datasets.
\end{abstract}